\documentclass{article}

\parskip = 2 pt
\oddsidemargin = 0 cm
\textwidth = 16 cm
\topmargin = -1 cm
\textheight = 24 cm

%%%%%%%%%%%%%%%%%%%%%%%%%%%%%%%%%%%%%%%%%%%%%%%%%%%%%%%%%%%%%%%%%%%%%%%%%
% MANPAGE like description setting for options, use as 
% \begin{Lentry} \item[text] text  \end{Lentry}

\usepackage{ifthen,calc}

\newcommand{\entrylabel}[1]{\mbox{\textsf{#1}}\hfil}
\newenvironment{entry}
  {\begin{list}{}
    {\renewcommand{\makelabel}{\entrylabel}
      \setlength{\labelwidth}{90pt}
      \setlength{\leftmargin}{\labelwidth+\labelsep}
    }
  }
  {\end{list}}

\newlength{\Mylen}
\newcommand{\Lentrylabel}[1]{%
  \settowidth{\Mylen}{\textsf{#1}}%
  \ifthenelse{\lengthtest{\Mylen > \labelwidth}}%
    {\parbox[b]{\labelwidth} %  term > labelwidth
      {\makebox[0pt][l]{\textsf{#1}}\\}} %
    {\textsf{#1}} %

  \hfil\relax}
\newenvironment{Lentry}
  {\renewcommand{\entrylabel}{\Lentrylabel}
   \begin{entry}}
  {\end{entry}}
%%%%%%%%%%%%%%%%%%%%%%%%%%%%%%%%%%%%%%%%%%%%%%%%%%%%%%%%%%%%%%%%%%%%%%%%%%

\begin{document}

\title{Boundary}
\author{Miguel Alcubierre, Gabrielle Allen, Gerd Lanfermann}
\date{1999}
\maketitle

\abstract{Standard boundary conditions with can be provided to
grid functions, or groups of grid functions. Available for 1D, 2D and
3D grid functions.}

\section{Purpose}

Allows you to apply standard boundary conditions to grid functions
or groups of grid functions on a Cartesian grid, taking into account 
a parallel decomposition of the grid. The routines are callable from 
C or Fortran. These routines are available for 1D, 2D and 3D grid functions,
and there are interfaces for applying boundary conditions in individual
directions or in all directions at once.

The boundary conditions available are
\begin{itemize}
\item Scalar
\item Flat 
\item Copy (static)
\item Radiation 
\item Robin
\end{itemize}

\vskip .5cm

\noindent
{\bf PERIODIC BOUNDARY CONDITIONS}: Periodic boundary conditions are currently
implemented by the driver in Cactus. See the documentation in 
{\bf CactusPUGH/PUGH} for more details.


\section{General Comments}
\begin{itemize}
\item{}
All the boundary conditions here take a stencil size as an 
argument. The stencil size is used to determine how many points
at the boundary should be updated with a boundary condition.
For example, a stencil size of two in each direction means
that the points at the boundary, as well as the points one in from the boundary
will be set by the boundary condition. These boundary points are part
of the total number of grid points that you have specified in the
beginning of the run.
\item{} 
Boundary routines can only be applied to grid functions. 
\item{}
All routines can be called by
%%% Subsitute with \begin{Lentry}
\begin{itemize}
\item{\em variable name}: ({\tt <implementation>:<var\_name> }) Suffix:
{\tt VN}; apply the boundary condition to the variable with the
specified name.
\item{\em group name}: ({\tt <implementation>:<group\_name>}) Suffix:
{\tt GN}; apply the boundary condition to all variables in the group.
\item{\em variable index}:  Suffix: {\tt VI}; apply the boundary
condition to the variable with the specified variable index.
\item{\em group index}:  Suffix: {\tt GI} apply the boundary
condition to all variables in the group with the specified group index.
\end{itemize}
\item{} For the boundary conditions in individual coordinate directions, 

\begin{tabular}{ll}
{\tt dir=-1} & apply at $x=x_{min}$ \\
{\tt dir=1} & apply at $x=x_{max}$ \\
{\tt dir=-2} & apply at $y=y_{min}$ \\
{\tt dir=2} & apply at $y=y_{max}$ \\
{\tt dir=-3} & apply at $z=z_{min}$ \\
{\tt dir=3} & apply at $z=z_{max}$ \\
\end{tabular} 

\end{itemize}

\subsection{Scalar Boundary Condition}

A scalar boundary condition means that the value of the given 
field or fields at the boundary is set to a given scalar value, for 
example zero. 

\subsubsection*{Calling from C:}

{\bf All Coordinate Directions:}
\begin{verbatim}
int ierr = BndScalarVN(cGH *cctkGH, int *stencil_size,  
                       CCTK_REAL var0, char *variable_name)
int ierr = BndScalarGN(cGH *cctkGH, int *stencil_size,  
                       CCTK_REAL var0, char *group_name)
int ierr = BndScalarVI(cGH *cctkGH, int *stencil_size,  
                       CCTK_REAL var0, int group_index)
int ierr = BndScalarGI(cGH *cctkGH, int *stencil_size,  
                       CCTK_REAL var0, int variable_index)
\end{verbatim}
{\bf Individual Coordinate Directions:}
\begin{verbatim}
int ierr = BndScalarDirVN(cGH *cctkGH, int stencil, int dir,
                       CCTK_REAL var0, char *variable_name)
int ierr = BndScalarDirGN(cGH *cctkGH, int stencil, int dir,
                       CCTK_REAL var0, char *group_name)
int ierr = BndScalarDirVI(cGH *cctkGH, int stencil, int dir,
                       CCTK_REAL var0, int group_index)
int ierr = BndScalarDirGI(cGH *cctkGH, int stencil, int dir,
                       CCTK_REAL var0, int variable_index)
\end{verbatim}


\subsubsection*{Calling from Fortran:}
{\bf All Coordinate Directions:}
\begin{verbatim}
call BndScalarVN(ierr, cctkGH, stencil_size, var0, variable_name)
call BndScalarGN(ierr, cctkGH, stencil_size, var0, group_name)
call BndScalarVI(ierr, cctkGH, stencil_size, var0, variable_index)
call BndScalarGI(ierr, cctkGH, stencil_size, var0, group_index)
\end{verbatim}
{\bf Individual Coordinate Directions:}
\begin{verbatim}
call BndScalarDirVN(ierr, cctkGH, dir, stencil, var0, variable_name)
call BndScalarDirGN(ierr, cctkGH, dir, stencil, var0, group_name)
call BndScalarDirVI(ierr, cctkGH, dir, stencil, var0, variable_index)
call BndScalarDirGI(ierr, cctkGH, dir, stencil, var0, group_index)
\end{verbatim}
where
{\tt
\begin{tabbing}
character*(*) \= variable\_name\=\kill
integer \> ierr \\
CCTK\_POINTER \> cctkGH\\
integer \> dir\\
integer \> stencil\\
integer \> stencil\_size(dim)\\
CCTK\_REAL \> var0 \\
character*(*) \> variable\_name\\
character*(*) \> group\_name\\
integer \> variable\_index\\
integer \> group\_index\\
\end{tabbing}
}

\subsubsection*{Arguments}
\begin{itemize}
\item[{\tt ierr}] Return value, negative value indicates the
boundary condition was not successfully applied
\item[{\tt cctkGH}] Grid hierarchy pointer
\item[{\tt var0}] Scalar value to apply  (For a complex grid function, this is the real part, the imaginary part is set to zero.)
\item[{\tt dir}] Coordinate direction in which to apply boundary condition
\item[{\tt stencil\_size}] Array with dimension of the grid function, containing the stencil width to apply the boundary at
\item[{\tt variable\_name}] Name of the variable
\item[{\tt group\_name}] Name of the group
\item[{\tt variable\_index}] Variable index
\item[{\tt group\_index}] Group index
\end{itemize}


\subsection{Flat Boundary Condition}

A flat boundary condition means that the value of the given 
field or fields at the boundary is copied from the value one grid point in,
in any direction. For example, for a stencil width of one, the
boundary value of phi {\tt phi(nx,j,k)}, on the positive x-boundary will
be copied from {\tt phi(nx-1,j,k)}. 

\subsubsection*{Calling from C:}

{\bf All Coordinate Directions:}
\begin{verbatim}
int ierr = BndFlatVN(cGH *cctkGH, int *stencil_size, char *variable_name)
int ierr = BndFlatGN(cGH *cctkGH, int *stencil_size, char *group_name)
int ierr = BndFlatVI(cGH *cctkGH, int *stencil_size, int variable_index)
int ierr = BndFlatGI(cGH *cctkGH, int *stencil_size, int group_index)
\end{verbatim}

{\bf Individual Coordinate Directions:}
\begin{verbatim}
int ierr = BndFlatDirVN(cGH *cctkGH, int stencil, int dir, char *variable_name)
int ierr = BndFlatDirGN(cGH *cctkGH, int stencil, int dir, char *group_name)
int ierr = BndFlatDirVI(cGH *cctkGH, int stencil, int dir, int variable_index)
int ierr = BndFlatDirGI(cGH *cctkGH, int stencil, int dir, int group_index)
\end{verbatim}

\subsubsection*{Calling from Fortran:}

{\bf All Coordinate Directions:}
\begin{verbatim}
call BndFlatVN(ierr, cctkGH, stencil_array, variable_name)
call BndFlatGN(ierr, cctkGH, stencil_array, group_name)
call BndFlatVI(ierr, cctkGH, stencil_array, variable_index)
call BndFlatGI(ierr, cctkGH, stencil_array, group_index)
\end{verbatim}

{\bf Individual Coordinate Directions:}
\begin{verbatim}
call BndFlatDirVN(ierr, cctkGH, stencil, dir, variable_name)
call BndFlatDirGN(ierr, cctkGH, stencil, dir, group_name)
call BndFlatDirVI(ierr, cctkGH, stencil, dir, variable_index)
call BndFlatDirGI(ierr, cctkGH, stencil, dir, group_index)
\end{verbatim}
where
{\tt
\begin{tabbing}
character*(*) \= variable\_name\=\kill
integer \> ierr \\
CCTK\_POINTER \> cctkGH\\
integer \> dir\\
integer \> stencil\\
integer \> stencil\_array(dim)\\
character*(*) \> variable\_name\\
character*(*) \> group\_name\\
integer \> variable\_index\\
integer \> group\_index\\
\end{tabbing}
}

\subsubsection*{Arguments}
\begin{itemize}
\item[{\tt ierr}] Return value, negative value indicates the
boundary condition was not successfully applied
\item[{\tt cctkGH}] Grid hierarchy pointer
\item[{\tt dir}] Coordinate direction in which to apply boundary condition
\item[{\tt stencil\_size}] Array with dimension of the grid function, containing the stencil width to apply the boundary at
\item[{\tt variable\_name}] Name of the variable
\item[{\tt group\_name}] Name of the group
\item[{\tt variable\_index}] Variable index
\item[{\tt group\_index}] Group index
\end{itemize}



\subsection{Radiation Boundary Condition}

This is a two level scheme. Grid functions are given for the current time 
level (to which the BC is applied) as well as grid functions from a past
timelevel which are needed for constructing the boundaray condition.
The grid function of the past time level need to have the same
geometry. When applying this boundary condition to a group, the
members of the group have to match up. Currently radiative boundary
conditions can only be applied with a stencil width of one in each
direction. 

The radiative boundary condition that is implemented is
\begin{equation}
\label{eqrad}
f = f_0 + \frac{u(r-vt)}{r}+\frac{h(r+vt)}{r}
\end{equation}
That is, outgoing radial waves with a 1/r
fall off, and the correct asymptotic value f0 are assumed, including
the possibility of incoming waves
(these incoming waves should be modeled somehow).

Condition~\ref{eqrad} above leads to the differential equation:
\begin{equation}
\frac{x^i}{r}\frac{\partial f}{\partial t}
+ v \frac{\partial f}{\partial x^i}
+\frac{v x^i}{r^2} (f-f_0)
= H \frac{v x^i}{r^2}  
\end{equation}
where $x^i$ is the normal direction to the given boundaries,
and $H = 2 dh(s)/ds$.

At a given boundary only the derivatives in the normal direction are 
considered.  Notice that $u(r-vt)$ has disappeared, but we still do 
not know the value of $H$.

To get $H$ we do the following:  The expression is evaluated one 
point in from the boundary and solved for $H$ there. Now we need a way of 
extrapolating $H$ to the boundary. For this, assume that 
$H$ falls off as a power law:
\begin{equation}
H = \frac{k}{r^n} \qquad \mbox{which gives} \qquad d_i H  =  - n \frac{H}{r}
\end{equation}
The value of $n$ is defined by the parameter {\tt radpower}.
If this parameter is negative, $H$ is forced to be zero (this
corresponds to pure outgoing waves and is the default).

The observed behaviour is the following:  Using $H=0$
is very stable, but has a very bad initial transient. Taking
$n$ to be 0 or positive improves the initial behaviour considerably,
but introduces a drift that can kill an evolution at very late
times.  Empirically, the best value found so far is $n=2$, for
which the initial behaviour is very nice, and the late time drift 
is quite small.

Another problem with this condition is that it does not
use the physical characteristic speed, but rather it assumes
a wave speed of $v$, so the boundaries should be out in
the region where the characteristic speed is constant.
Notice that this speed does not have to be 1.  

\subsubsection*{Calling from C:}

{\bf All Coordinate Directions:}
\begin{verbatim}
int ierr = BndRadiativeVN(cGH *cctkGH, int *stencil_size, 
                          CCTK_REAL limit, CCTK_REAL speed, 
                          char *variable_name, char *variable_name_past)
int ierr = BndRadiativeGN(cGH *cctkGH, int *stencil_size, 
                          CCTK_REAL limit, CCTK_REAL speed,  
                          char *group_name, char *group_name_past)
int ierr = BndRadiativeVI(cGH *cctkGH, int *stencil_size, 
                          CCTK_REAL limit, CCTK_REAL speed,  
                          int variable_index, int variable_index_past)
int ierr = BndRadiativeGI(cGH *cctkGH, int *stencil_size, 
                          CCTK_REAL limit, CCTK_REAL speed,  
                          int group_index, int group_index_past)
\end{verbatim}

{\bf Individual Coordinate Directions:}
\begin{verbatim}
int ierr = BndRadiativeDirVN(cGH *cctkGH, int stencil, int dir,
                             CCTK_REAL limit, CCTK_REAL speed, 
                             char *variable_name, char *variable_name_past)
int ierr = BndRadiativeDirGN(cGH *cctkGH, int *stencil, int dir, 
                             CCTK_REAL limit, CCTK_REAL speed,  
                             char *group_name, char *group_name_past)
int ierr = BndRadiativeDirVI(cGH *cctkGH, int *stencil, int dir, 
                             CCTK_REAL limit, CCTK_REAL speed,  
                             int variable_index, int variable_index_past)
int ierr = BndRadiativeDirGI(cGH *cctkGH, int *stencil, int dir, 
                             CCTK_REAL limit, CCTK_REAL speed,  
                             int group_index, int group_index_past)
\end{verbatim}

\subsubsection*{Calling from Fortran:}

{\bf All Coordinate Directions:}
\begin{verbatim}
call BndRadiativeVN(ierr, cctkGH, stencil_size, speed, limit, 
                    variable_name, variable_name_past)
call BndRadiativeVN(ierr, cctkGH, stencil_size, speed, limit, 
                    group_name, group_name_past)
call BndRadiativeVN(ierr, cctkGH, stencil_size, speed, limit, 
                    variable_index, variable_index_past)
call BndRadiativeVN(ierr, cctkGH, stencil_size, speed, limit, 
                    group_index, group_index_past)
\end{verbatim}

{\bf Individual Coordinate Directions:}
\begin{verbatim}
call BndRadiativeDirVN(ierr, cctkGH, stencil, dir, speed, limit, 
                       variable_name, variable_name_past)
call BndRadiativeDirVN(ierr, cctkGH, stencil, dir, speed, limit, 
                       group_name, group_name_past)
call BndRadiativeDirVN(ierr, cctkGH, stencil, dir, speed, limit, 
                       variable_index, variable_index_past)
call BndRadiativeDirVN(ierr, cctkGH, stencil, dir, speed, limit, 
                       group_index, group_index_past)
\end{verbatim}
where
{\tt
\begin{tabbing}
character*(*) \= variable\_name\=\kill
integer \> ierr \\
CCTK\_POINTER \> cctkGH\\
integer \> dir\\
integer \> stencil\\
integer \> stencil\_array(dim)\\
character*(*) \> variable\_name\\
character*(*) \> group\_name\\
integer \> variable\_index\\
integer \> group\_index\\
CCTK\_REAL\>speed\\
CCTK\_REAL\>limit\\
\end{tabbing}
}




where
\begin{itemize}
\item[{\tt integer ierr}] return value, operation failed when return
value {\em negative}
\item[{\tt CCTK\_POINTER cctkGH}] grid hierarchy pointer
\item[{\tt integer stencil\_size(dim)}] array of size {\tt dim}
(dimension of the gridfunction). To how many points from the outer
boundary to apply the boundary condition. 
\item[{\tt CCTK\_REAL speed}] wave speed used for boundary condition ($v$).
	
\item[{\tt CCTK\_REAL limit}] asymptotic value of function at infinity

\item[{\tt character*(*) variable\_name}] the name of the grid function 
	to which the boundary condition will be applied
\item[{\tt character*(*) variable\_name\_past}]  
    The name of the grid function 
    containing the values on the past time level, needed to calculate
    the boundary condition.

\item[{\tt character*(*) group\_name}] the name of the group
	to which the boundary condition will be applied
\item[{\tt character*(*) group\_name\_past}] is the name of the group
    containing the grid functions on the past time level, needed to calculate
    the boundary condition.

\item[{\tt integer variable\_index}] the index of the grid function 
	to which the boundary condition will be applied
\item[{\tt integer variable\_index\_past}] the index of the grid function 
    containing the values on the past time level, needed to calculate
    the boundary condition.

\item[{\tt integer group\_index}] the index of the group
	to which the boundary condition will be applied
\item[{\tt integer group\_index\_past}] the index of the group
    containing the values on the past time level, needed to calculate
    the boundary condition.
\end{itemize}


\subsection{Copy Boundary Condition}

This is a two level scheme. Copy the boundary values from a different
grid function, for example the previous timelevel. The two grid functions
(or groups of grid functions) must have the same geometry.

\subsubsection*{Calling from C:}
\begin{verbatim}
int ierr = BndCopyVN(cGH *cctkGH, int *stencil_size, 
                     char *variable_name_to, char *variable_name_from)
int ierr = BndCopyGN(cGH *cctkGH, int *stencil_size, 
                     char *group_name_to, char *group_name_from)
int ierr = BndCopyVI(cGH *cctkGH, int *stencil_size, 
                     int variable_index_to, int variable_index_from)
int ierr = BndCopyGI(cGH *cctkGH, int *stencil_size, 
                     int group_index_to, int group_index_from)
\end{verbatim}

\subsubsection*{Calling from Fortran:}
\begin{verbatim}
call BndCopyVN(ierr, cctkGH, stencil_size, variable_name_to,  
               variable_name_from)
call BndCopyVN(ierr, cctkGH, stencil_size, group_name_to,     
               group_name_from)
call BndCopyVN(ierr, cctkGH, stencil_size, variable_index_to, 
               variable_index_from)
call BndCopyVN(ierr, cctkGH, stencil_size, group_index_to,    
               group_index_from)
\end{verbatim}
where
\begin{Lentry}
\item[{\tt integer ierr}] return value, operation failed when return
value {\em negative}
\item[{\tt CCTK\_POINTER cctkGH}] grid hierarchy pointer
\item[{\tt integer stencil\_size(dim)}] array of size {\tt dim}
(dimension of the gridfunction). To how many points from the outer
boundary to apply the boundary condition. 

\item[{\tt character*(*) variable\_name\_to}] the name of the grid function 
	to which the boundary condition will be applied by copying to.
\item[{\tt character*(*) variable\_name\_from}]  is the name of the grid function 
    containing the values to copy from.

\item[{\tt character*(*) group\_name\_to}] the name of the group
	to which the boundary condition will be applied by copying to.
\item[{\tt character*(*) group\_name\_from}] is the name of the group
    containing the the values to copy from.

\item[{\tt integer variable\_index\_to}] the index of the grid function 
	to which the boundary condition will be applied by copying to.
\item[{\tt integer variable\_index\_from}] the index of the grid function 
    containing the the values to copy from.

\item[{\tt integer group\_index\_to}] the index of the group
	to which the boundary condition will be applied by copying to.
\item[{\tt integer group\_index\_from}] the index of the group
    containing the the values to copy from.
\end{Lentry}


\subsection{Robin Boundary Condition}

This boundary condition has not yet been implemented in 
individual coordinate directions.
The Robin boundary condition is:
\begin{equation}
f(r) = f_0 + \frac{k}{r^n}
\end{equation}
with $k$ a constant, $n$ the decay rate and $f_0$ the value at infinity. This implies:
\begin{equation}
\frac{\partial f}{\partial r} =  - n \frac{k}{r^{n+1}}
\end{equation}
or
\begin{equation}
\frac{\partial f}{\partial r} = - n \frac{(f-f_0)}{r}
\end{equation}
Considering now a given cartesian direction $x$  we get:
\begin{equation}
\frac{\partial f}{\partial x} =
\frac{\partial f}{\partial r} 
\frac{\partial r}{\partial x} = \frac{x}{r}\frac{\partial f}{\partial r}
\end{equation}
which implies
\begin{equation}
\frac{\partial f}{\partial x} = - n (f-f_0)\frac{x}{r^2}
\end{equation}
The equations are then finite differenced around the grid point $i+1/2$:
\begin{equation}
f_{i+1} - f_i = - n \Delta x \left( \frac{1}{2}(f_{i+1}+f_i) - f_0\right) \frac{x_{i+1/2}}{r^2_{i+1/2}}
\end{equation}
or
\begin{equation}
f_{i+1}-f_i = -n \Delta x ( (f_{i+1}+f_i)-2 f_0)\frac{x_{i+1}+x_i}{(r_{i+1}+r_i)^2}
\end{equation}
And this is then solved either for $f_i$ or $f_{i+1}$ depending on which side are
we looking at. 


\subsubsection*{Calling from C:}

{\bf All Coordinate Directions:}
\begin{verbatim}
int ierr = BndRobinVN(cGH *cctkGH, int *stencil_size,  
                       CCTK_REAL finf, int npow, char *variable_name)
int ierr = BndScalarGN(cGH *cctkGH, int *stencil_size,  
                       CCTK_REAL finf, int npow, char *group_name)
int ierr = BndScalarVI(cGH *cctkGH, int *stencil_size,  
                       CCTK_REAL finf, int npow, int group_index)
int ierr = BndScalarGI(cGH *cctkGH, int *stencil_size,  
                       CCTK_REAL finf, int npow, int variable_index)
\end{verbatim}


\subsubsection*{Calling from Fortran:}
{\bf All Coordinate Directions:}
\begin{verbatim}
call BndRobinVN(ierr, cctkGH, stencil_size, finf, npow, variable_name)
call BndRobinGN(ierr, cctkGH, stencil_size, finf, npow, group_name)
call BndRobinVI(ierr, cctkGH, stencil_size, finf, npow, variable_index)
call BndRobinGI(ierr, cctkGH, stencil_size, finf, npow, group_index)
\end{verbatim}
where
{\tt
\begin{tabbing}
character*(*) \= variable\_name\=\kill
integer \> ierr \\
CCTK\_POINTER \> cctkGH\\
integer \> stencil\_size(dim)\\
CCTK\_REAL \> finf \\
integer \> npow \\
character*(*) \> variable\_name\\
character*(*) \> group\_name\\
integer \> variable\_index\\
integer \> group\_index\\
\end{tabbing}
}

\subsubsection*{Arguments}
\begin{Lentry}
\item[{\tt ierr}] Return value, negative value indicates the
boundary condition was not successfully applied
\item[{\tt cctkGH}] Grid hierarchy pointer
\item[{\tt finf}] Scalar value at infinity
\item[{\tt npow}] Decay rate ($n$ in discussion above)
\item[{\tt stencil\_size}] Array with dimension of the grid function, containing the stencil width to apply the boundary at
\item[{\tt variable\_name}] Name of the variable
\item[{\tt group\_name}] Name of the group
\item[{\tt variable\_index}] Variable index
\item[{\tt group\_index}] Group index
\end{Lentry}


% Automatically created from the ccl files by using gmake thorndoc
\include{interface}
\include{param}
\include{schedule}

\end{document}
