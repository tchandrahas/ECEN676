% Thorn documentation template
\documentclass{article}
\begin{document}

\title{IOBasic}
\author{Gabrielle Allen, Thomas Radke}
\maketitle

\abstract{Thorn IOBasic provides I/O methods for outputting scalar values in
ASCII format into files and for printing them as runtime information to screen.}

\section{Purpose}
Thorn IOBasic registers 2 I/O methods with the I/O interface in the flesh
which both output the same following information:
%
\begin{itemize}
  \item for CCTK\_SCALAR variables, the value of the scalar versus time
  \item for CCTK\_GF and CCTK\_ARRAY variables, the values of global reduction
        operations (eg. minimum, maximum, L1, and L2 norm) versus time
\end{itemize}
%
The I/O methods differ in the destination the output is written to:
\begin{itemize}
  \item{\tt Scalar}\\
    This method outputs the information into ASCII files named {\tt "<scalar\_name>.tl"} (for CCTK\_SCALAR variables) and {\tt "<var\_name>\_<reduction>.tl"}
    (for CCTK\_GF and CCTK\_ARRAY variables where {\tt reduction} would stand
    for {\tt min, max, nm1, nm2} respectively).\\
    The output data can be plotted by using either {\it xgraph} or
    {\it gnuplot}. The output style can be selected via parameter settings.
  \item {\tt Info}\\
    This method prints the data as runtime information to {\it stdout}.\\
    The output occurs as a table with columns containing the current iteration
    number, the physical time at this iteration, and more columns for
    scalar/reduction values of each variable to be output.\\
\end{itemize}
%
%
\section{{\tt IOBasic} Parameters}
%
Parameters to control the {\tt Scalar} I/O method are:
\begin{itemize}
  \item {\tt IOBasic::outScalar\_every} (steerable)\\
        How often to do info output. If this parameter is set in the
        parameter file, it will override the setting of the shared
        {\tt IO::out\_every} parameter.
  \item {\tt IOBasic::outdirScalar}\\
        The directory in which to place the ASCII output files.\\
        If the directory doesn't exist at startup it will be created.
  \item {\tt IOBasic::outScalar\_style}\\
        How to start comments in the ASCII output files.\\
        Possible choices for this keywork parameter are {\it xgraph} and
        {\it gnuplot}.
  \item {\tt IOBasic::out\_format} (steerable)\\
        The output format for floating-point numbers in {\tt Scalar} output.\\
        This parameter conforms to the format modifier of the C library routine
        {\it fprintf(3)}. You can set the format for outputting floating-point
        numbers (fixed or exponential) as well as their precision (number of
        digits).
  \item {\tt IOBasic::outScalar\_vars} (steerable)\\
        The list of variables to output into individual ASCII files.\\
        The variables must be given by their fully qualified variable or group
        name. The special keyword {\it all} requests info output for all
        variables. Multiple variables must be separated by spaces.
  \item {\tt IOBasic::outScalar\_reductions} (steerable)\\
        The list of global reduction operations to perform on
        CCTK\_GF and CCTK\_ARRAY variables for {\tt Scalar} output.\\
        Multiple reduction names must be separated by spaces.
\end{itemize}
%
%
Parameters to control the {\tt Info} I/O method are:
\begin{itemize}
  \item {\tt IOBasic::outInfo\_every} (steerable)\\
        How often to do info output. If this parameter is set in the
        parameter file, it will override the setting of the shared
        {\tt IO::out\_every} parameter.
  \item {\tt IOBasic::outInfo\_vars} (steerable)\\
        The list of variables to output to screen.\\
        The variables must be given by their fully qualified variable or group
        name. The special keyword {\it all} requests info output for all
        variables. Multiple variables must be separated by spaces.\\
        For CCTK\_GF and CCTK\_ARRAY variables, an option string can be appended
        in square brackets to the name of the variable. The only option
        supported so far is an individual list of reductions for that variable
        which would take precedence over the default reduction operations to
        perform.
  \item {\tt IOBasic::outInfo\_reductions} (steerable)\\
        The default list of global reduction operations to perform on
        CCTK\_GF and CCTK\_ARRAY variables. This setting can be overridden
        for individual variables using an option string.\\
        Multiple reduction names must be separated by spaces.
\end{itemize}
All of the above parameters marked as steerable can be changed at runtime.
%
%
\section{Examples}
%
{\bf Example for {\tt Info} Output}\\
%
The following parameter settings request info output for variables {\tt
grid::r, wavetoy::phi} (both are CCTK grid functions) and {\tt mythorn::complex}
(a complex CCTK scalar) at every other iteration.

The minimum and maximum of {\tt grid::r} is printed according to the list
of default reductions for info output (parameter {\tt IOBasic::outInfo\_reductions}). This list is overridden for {\tt wavetoy::phi} where only the L2 norm
is output as specified in the option string for this variable. You can also
add other reduction operators within the $<>$ braces.

For the scalar variable {\tt mythorn::complex} both the real and imaginary part are printed.
\begin{verbatim}
IOBasic::outInfo_every      = 2
IOBasic::outInfo_vars       = "grid::r
                               wavetoy::phi[reductions=<norm2>]
                               mythorn::complex"
IOBasic::outInfo_reductions = "minimum maximum"
\end{verbatim}
\vspace*{2ex}
The resulting screen output would look like this:
\begin{tiny}
\begin{verbatim}
---------------------------------------------------------------------------------------------
  it  |          | GRID::r                     | WAVETOY::phi | MYTHORN::complex            |
      |    t     | minimum      | maximum      | norm2        | real part    | imag part    |
---------------------------------------------------------------------------------------------
    0 |    0.000 |   0.02986294 |   0.86602540 |   0.04217014 |   6.90359593 |   0.00000000 |
    2 |    0.034 |   0.02986294 |   0.86602540 |   0.00934749 |   6.90359593 |   0.00000000 |
    4 |    0.069 |   0.02986294 |   0.86602540 |   0.02989811 |   6.90359593 |   0.00000000 |
    6 |    0.103 |   0.02986294 |   0.86602540 |   0.05899959 |   6.90359593 |   0.00000000 |
    8 |    0.138 |   0.02986294 |   0.86602540 |   0.07351147 |   6.90359593 |   0.00000000 |
   10 |    0.172 |   0.02986294 |   0.86602540 |   0.07781795 |   6.90359593 |   0.00000000 |
\end{verbatim}
\end{tiny}
%
\vspace*{3ex}
{\bf Example for {\tt Scalar} Output}\\
%
The following parameter settings request scalar output for all grid function
variables in the group {\tt grid::coordinates} and for the scalar variable
{\tt grid::coarse\_dx}.\\
Output occurs every 10th iteration. {\tt gnuplot} output style is selected
for the ASCII files which are placed into a subdirectory {\tt scalar\_output}.
\begin{verbatim}
IOBasic::outScalar_every      = 10
IOBasic::outScalar_vars       = "grid::coordinates grid::coarse_dx"
IOBasic::outScalar_reductions = "minimum maximum"
IOBasic::outScalar_style      = "gnuplot"
IOBasic::outdirScalar         = "scalar_output"
\end{verbatim}
\vspace*{2ex}
This would create the following ASCII files:
\begin{verbatim}
~/Cactus/par> ls scalar_output
coarse_dx.tl  r_min.tl  x_min.tl  y_min.tl  z_min.tl
r_max.tl      x_max.tl  y_max.tl  z_max.tl
\end{verbatim}
%
%
\section{Comments}
%
{\bf Possible Reduction Operations}\\
%
In order to get output of reduction values for {\tt CCTK\_GF} and {\tt
CCTK\_ARRAY} variables you need to activate a thorn which provides
reduction operators (eg. thorn {\tt PUGHSlab} in the {\tt CactusPUGH}
arrangement). For a list of possible reduction operations please refer to
the documention of this reduction thorn.\\[3ex]
%
{\bf Getting Output from IOBasic's I/O Mehtods}\\
%
You obtain output by an I/O method by either
%
\begin{itemize}
  \item setting the appropriate I/O parameters
  \item calling one of the routines of the I/O function interface
        provided by the flesh
\end{itemize}
%
For a description of basic I/O parameters and the I/O function interface to
invoke I/O methods by application thorns please see the documentation of thorn
{\tt IOUtil} and the flesh.\\[3ex]
%
%
{\bf Building Cactus configurations with IOBasic}\\
%
Since {\tt IOBasic} uses parameters from {\tt IOUtil} it also needs this I/O
helper thorn be compiled into Cactus and activated at runtime in the
{\tt ActiveThorns} parameter in your parameter file.
%
% Automatically created from the ccl files 
% Do not worry for now.
\include{interface}
\include{param}
\include{schedule}

\end{document}
