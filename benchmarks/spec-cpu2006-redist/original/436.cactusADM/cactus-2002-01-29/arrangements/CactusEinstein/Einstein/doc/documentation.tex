\documentclass{article}
\usepackage{latexsym}

\begin{document}

%%%%%%%%%%%%%%%%%%%%%%%%%%%%%%%%%%%%%%%%%%%%%%%%%%%%%%%%%%%%%%%%%%%%%%%%%%%%%%%%
%%%%%%%%%%%%%%%%%%%%%%%%%%%%%%%%%%%%%%%%%%%%%%%%%%%%%%%%%%%%%%%%%%%%%%%%%%%%%%%%
%%%%%%%%%%%%%%%%%%%%%%%%%%%%%%%%%%%%%%%%%%%%%%%%%%%%%%%%%%%%%%%%%%%%%%%%%%%%%%%%

\title{Einstein}
\author{AEI/NCSA/WashU Relativity Groups}
\date{1997-1999\\$ $Id: documentation.tex,v 1.6 2001/09/27 11:12:33 lanfer Exp $ $}
\maketitle

\abstract{Base thorn for Einstein arrangement}

%%%%%%%%%%%%%%%%%%%%%%%%%%%%%%%%%%%%%%%%%%%%%%%%%%%%%%%%%%%%%%%%%%%%%%%%%%%%%%%%

\section{Purpose}
The Einstein thorn defines the standard relativity variables and
parameters, in addition to a number of scheduled and utility routines
relevant for any thorn using them.

%%%%%%%%%%%%%%%%%%%%%%%%%%%%%%%%%%%%%%%%%%%%%%%%%%%%%%%%%%%%%%%%%%%%%%%%%%%%%%%%

\section{Grid Functions}

Einstein defines 3D grid functions for the 
\begin{itemize}
\item	conformal or physical 3-metric, depending on the
	{\tt conformal\_state} grid scalar described below
	($g_{ij}$ or $\hat{g}_{ij} = $\{{\tt gxx,gxy,gxz,gyy,gyz,gzz}\}),
\item	extrinsic curvature ($K_{ij} = $\{{\tt kxx,kxy,kxz,kyy,kyz,kzz}\}), 
\item	time-independent conformal factor ($\Psi = ${\tt psi}),
\item	spatial 1st and 2nd derivatives of the conformal factor \\
	($\Psi_{,i} = $\{{\tt psix,psiy,psiz}\}), \\
	($\Psi_{,ij}= $\{{\tt psixx,psixy,psixz,psiyy,psiyz,psizz}\}).
\item	lapse function ($\alpha=${\tt alp}) 
\item	shift function ($\beta^i= $\{{\tt betax,betay,betaz}\})
\end{itemize}

The conformal 3-metric $g_{ij}$ is related to the physical 3-metric
$\hat{g}_{ij}$ by 
$$
\hat{g}_{ij}(t,x^i) = \Psi^4(x^i) g_{ij}(t,x^i)
$$
The grid scalar {\tt conformal\_state} is used to keep track of
what the 3-metric grid functions \{{\tt gxx,gxy,gxz,gyy,gyz,gzz}\}
mean.  {\tt conformal\_state} should be set to one of the following
(integer) constants defined in the header file {\tt Einstein.h}:
\begin{description}
\item[{\tt CONFORMAL\_METRIC}]
	$\Rightarrow$ the grid functions are the conformal 3-metric
	$g_{ij}$.
\item[{\tt NOCONFORMAL\_METRIC}]
	$\Rightarrow$ the grid functions are the physical 3-metric
	$\hat{g}_{ij}$.
\end{description}

The utility routines
\begin{verbatim}
   ConfToPhys(nx,ny,nz, psi, gxx,gxy,gxz,gyy,gyz,gzz)
   PhysToConf(nx,ny,nz, psi, gxx,gxy,gxz,gyy,gyz,gzz)
\end{verbatim}
may be used to change the values of the 3-metric grid functions
from one to the other everywhere in the grid, i.e.{} to update the
3-metric grid functions via ${\tt g}ij \leftarrow \Psi^{\pm 4} {\tt g}ij$
everywhere in the grid.%%%
\footnote{%%%
	 Strictly speaking, these routines actually update
	 the grid functions in this way only for the current
	 processor's chunk of the grid; you call these
	 routines on each processor to update the grid
	 functions everywhere in the grid.%%%
	 }%%%
{}  You may call these routines as often as you like to switch the
3-metric grid functions back and forth.  Note that these routines
do {\em not\/} alter {\tt conformal\_state}; you must to keep this
up-to-date yourself!

In theory all thorns which use the 3-metric should honor honor
{\tt conformal\_state}, i.e. it should be ok to invoke any thorn
with {\tt conformal\_state} and the grid functions in either state.
But to be on the safe side, it's probably wise for thorns to save
the initial value of {\tt conformal\_state}, and restore things to
that state before they return to the flesh.

Einstein provides analysis routines to calculate the
\begin{itemize}
\item	trace of the extrinsic curvature ({\tt trK})
\item	determinant of the conformal 3-metric ({\tt detg}),
\item	components of the conformal 3-metric in 
	spherical polar coordinates ({\tt grr,grp,grq,gpp,gqp,gqq})
\item	components of the extrinsic curvature in 
	spherical polar coordinates ({\tt krr,krp,krq,kpp,kqp,kqq})
\end{itemize}

There are also utility routines to register slicing conditions;
these are discussed in detail in sections~\ref{sect-slicing-conditions}
and~\ref{sect-programming-details}.

%%%%%%%%%%%%%%%%%%%%%%%%%%%%%%%%%%%%%%%%%%%%%%%%%%%%%%%%%%%%%%%%%%%%%%%%%%%%%%%%

\section{Comments}

Activating the Einstein thorn provides storage for the metric,
extrinsic curvature, lapse and shift grid functions. Storage
for the shift and conformal factor variables is assigned 
depending on the parameters {\tt use\_shift} and {\tt use\_conformal}.

%%%%%%%%%%%%%%%%%%%%%%%%%%%%%%%%%%%%%%%%%%%%%%%%%%%%%%%%%%%%%%%%%%%%%%%%%%%%%%%%

\section{Slicing Conditions}
\label{sect-slicing-conditions}

Einstein does not itself implement any slicing conditions
however it does hold the parameter 
{\tt slicing} which other thorns may extend.
Thorns extending this parameter should
respect the following definitions for slicings

\begin{itemize}

\item{\tt geodesic}

This specifies so-called "geodesic" slicing, which corresponds 
to keeping the lapse fixed to one for the whole evolution. It is called geodesic slicing
because one can prove that in this case, the world-lines of the observers that move
normal to the spatial hypersurfaces are geodesics. This slicing
is the standard one when one considers non-relativistic theories.
In General Relativity however, it is well known to be generally a bad choice, 
since obdervers moving along geodesics (i.e. in free fall) can
easily focus and collide. Whenever this happens our slicing
becomes singular, and the code will crash.

Geodesic slicing can be very useful for testing the evolution in
some limited regimes.

\item{\tt static}

This condition is similar to {\tt geodesic} slicing in the
fact that the lapse does not evolve with time, but now it is allowed
to have an initial profile that is different from one. This
initial profile is controlled by the parameter 
{\tt initial\_lapse} which will be discussed below.

\item{\tt power\_law}

This generalised slicing condition includes as special cases
harmonic, 1+log, detg and the Bona-Masso family of slicings. 
The general form is
$$
\partial_t \alpha = - \alpha^n f {\mbox tr} K
$$
where ${\mbox tr}K$ is the trace of the extrinsic curvature tensor,
and $f$ is a real positive constant.

\item{\tt harmonic}

Harmonic slicing is a just {\tt power\_law} slicing with {\tt n}=2.
The name harmonic slicing comes from the fact that it 
corresponds to a harmonic time coordinate,
i.e. one that satisfies the equation
$$
\Box t = 0
$$

\item{\tt 1+log}

This "generalised 1+log slicing'' corresponds to
{\tt power\_law} with ${\tt n}=1$. The name {\tt 1+log}
comes from the fact that one can integrate
the evolution equation for the lapse to find that
$$
\alpha = H(x^i) + \log(g^{f/2})
$$
with $H$ a function of space only (in many cases taken to be one),
and $g$ the determinant of the spatial metric (the `{\it volume
element}).


\item{\tt maximal}

Maximal slicing corresponds to the condition
$$
{\mbox tr}K = \partial_t {\mbox tr} K = 0
$$
which leads to the following elliptic equation for the lapse
$$
\nabla^2\alpha = \alpha K_{ij} K^{ij}
$$

Since the lapse is solved via an elliptic equation, no initial lapse
profile is necessary.
All that one needs to specify is a boundary condition and elliptic solver.

%%%%%%%%%%%%%%%%%%%%%%%%%%%%%%%%%%%%%%%%%%%%%%%%%%%%%%%%%%%%%%%%%%%%%%%%%%%%%%%%

\section{Courant Condition}

Einstein provides routines for calculating the wave speeds on the grid,
so that a Courant condition for the timestep can be implemented by
thorn {\tt CactusBase/Time}. This can only be used for generalised harmonic 
or generalised 1+log slicings. For these slicing, the characteristic
wave speeds on the grid, in the direction $x^i$ are
$$
v_p^i = \alpha \sqrt(g^{ii})-\beta^i
$$
for the physical degrees of freedom, and
$$
v_g^i = \alpha f \sqrt(g^{ii})-\beta^i
$$
for the gauge degrees of freedom. Here $g^{ii}$ refers to the $xx$, $yy$ or $zz$ component of
the {\it physical} metric, and $f$ is the parameter {\tt harmonic\_f}.
 The maximum wavespeed is then the maximum of all these speeds,
over the grid, over all directions. This is the value of {\tt courant\_wave\_speed} which
is passed to thorn {\tt CactusBase/Time}.  We also need to calculate the minumum time for
a wave to cross a zone, passed as {\tt courant\_time} to {\tt CactusBase/Time}. This is
given by the minimum of $\Delta x^i/v_p^i$ and $\Delta x^i/v_g^i$. 

\end{itemize}

%%%%%%%%%%%%%%%%%%%%%%%%%%%%%%%%%%%%%%%%%%%%%%%%%%%%%%%%%%%%%%%%%%%%%%%%%%%%%%%%

\section{Programming Details}
\label{sect-programming-details}

\subsection{Slicing registration}
To allow for a flexible and expandable way of conditionally executing slicing
routines we integrated a slicing registration structure into the Einstein
thorn. The slicing registry works this way:
We assume an evolution routine that offers some slicing algorithm as
part of its evolution routine and an external thorn's slicing algorithm, e.g.
maximal slicing. Each of these slicing is registered by name and
obtains a unique number (``{\em handle}''). Before the evolution routine is
carried out, the slicing parameters (and conditional functions (see
below)) are evaluted and a (global) grid scalar is set to the handle
of the slicing which is {\em active} during the next iterative step.

Within the evolution routine, the code has to check if it is time to
execute any of the slicings it offers or make a pass. In the code, the 
programmer gets the handle for a particular slicing and compares it
to {\tt active\_slicing\_handle}. It it matches, it can execute the
slicing code, otherwise he continues, comparing to the next slicing he offers.

Two parameters are used by the slicing mechanism: {\tt slicing}, which 
can hold {\bf one} slicing name or the keyword {\tt mixed}. If {\tt
mixed} is set, the second parameter {\tt mixed\_slicing} may be used to 
specify more than one slicing, where priorities decrease with order/position.
The use of conditional slicing functions is explained in the next section.


\subsection{Using slicing registration}
Here we give a detailed programming guide on how to incorporate a slicing and 
a conditional function.
\begin{enumerate}
\item{\bf active\_slicing\_handle}: this grid scalar can be understood
as global integer, that takes the handle of next active slicing. It is 
set by the routine {\tt  Einstein\_SetNextSlicing}, scheduled at
CCTK\_PRESTEP. This routine evaluates {\tt slicing}, {\tt
mixed\_slicing} and any conditional slicing functions registered
for any of the slicings mentioned in  {\tt mixed\_slicing}. Note that
you cannot have a conditional slicing function for the single slicing mentioned 
in the parameter {\tt slicing}.

\item{\bf conditional slicing function} this is a function which is
registered together with a slicing. This function must return the values
{\tt SLICING\_YES}, {\tt SLICING\_DONTCARE} or {\tt SLICING\_NO}, (defined
in {\tt ./Einstein/src/Slicing.h}). Depending of these return value, a 
slicing is ``activated'': its handle is assigned to {\em
active\_slicing\_handle}. Should more than one slicing insists on being
activated, the slicing which is mentioned first in {\tt mixes\_slicing} 
is given priority.

\item{\bf Register slicing}: Initially, a thorn or evolution routine
registers the slicing it offers with the slicing database {\em by name}.
A index number (``{\em handle}'')is assigned to each of the
slicings. The name under which the slicing is referenced in the
database {\em has to be the same} as the value taken by the parameter
{\tt slicing} ( same applies to{\tt mixed\_slicing}). For example in {\tt param.ccl}: 
\begin{verbatim}
shares:einstein

EXTENDS KEYWORD slicing ""
{
   "static"   :: "Lapse is not evolved"
   "geodesic" :: "Lapse is set to unity"
}
\end{verbatim}
These slicing can now be registered with the slicing database in {\tt Startup.c}

\begin{verbatim}
#include "CactusEinstein/Einstein/src/Slicing.h"

void MyEvol_RegisterSlicing(void) {
  DECLARE_CCTK_PARAMETERS
  handle = Einstein_RegisterSlicing("geodesic");
  handle = Einstein_RegisterSlicing("static");
}
\end{verbatim}
The function {\tt MyEvol\_RegisterSlicing(void)} can be scheduled at CCTK\_INITIAL
FIXME. In this example,  the handle of the slicing is returned by the
registration call but
not used any further. Note that the slicing has to be coded, there is
no check by system if this slicing code exists.

\item{\bf Register ``time-to-slice'' function}: in addition, a
conditional function can be associated with the slicing handle and 
will be evalutated everytime that slicing is activated. This allows
for simple conditionals (``use this slicing every second iteration'')
to rather complex slicing conditionals (``use this slicing, when the
lapse has collapsed in the center''). An extension of the example
above could read:
\begin{verbatim}
 
#include "CactusEinstein/Einstein/src/Slicing.h"

void MyEvol_RegisterSlicing(void) {
  DECLARE_CCTK_PARAMETERS

  int LapseHasCollapsed(cGH *GH);
  int SliceEverySecondIteration(cGH *GH);

  handle = Einstein_RegisterSlicing("default_slice");

  handle = Einstein_RegisterSlicing("geodesic");
  active = Einstein_RegisterTimeToSlice(handle, SliceEverySecondIteration);	

  handle = Einstein_RegisterSlicing("static");
  active = Einstein_RegisterTimeToSlice(handle, LapseHasCollapsed);	

}


int SliceEverySoManyIteration(cGH *GH) {
  DECLARE_PARAMETER
  if (cGH->cctk_iteration%geodesic_every==0) 
    return(SLICING_YES);
  else 
    return(SLICING_NO);
}

int LapseHasCollapsed(cGH *GH) {
  if (CheckLapse(GH)) 
    return(SLICING_YES)
  else
    return(SLICING_DONTCARE)
}

\end{verbatim}
Note that we include the Slicing.h, which defines our return values
for the conditional functions. 

In the example above we register two
slicings and combine them with a condition:
Slicing {\em geodesic} is activated only if the associated function
{\tt SliceEverySoManyIteration} returns {\tt SLICING\_YES}. (This is
the case if the iteration number modulo the value set in the parameter 
{\tt geodesic\_every} is zero.)
Otherwise, this slicing is {\em not} executed and the default slicing
has to kick in. This default slicing could 
be a slicing that has either {\em no} conditional function or its
conditional function returns a {\tt SLICING\_DONTCARE}.

Slicing {\em static} is activated, if the associated function returns
{\tt SLICING\_YES}, which is the case if the lapse has collapsed 
(as signaled by {\tt CheckLapse(GH)}). If the associated function returns
the value {\tt SLICING\_DONTCARE} for a failed laps check,
this slicing will be executed if no other slicing insists on being
activated. 


If two slicings insist on being activated ({\tt SLICING\_YES}) or two
slicings don't care ({\tt SLICING\_DONTCARE}), the slicing mentioned in
the parameter {\tt mixed\_slicing} is given priority. 

\item{\bf Using the parameters}
For the example above, the slicing section of the parameter file could 
look like this:
\begin{verbatim}
einstein::slicing         =  ``mixed''
einstein::mixed_slicing   =  ``default_slice geodesic static''
einstein::geodesic_every  =  2
\end{verbatim}
Here we would execute {\em default\_slice} by default and+ every second
iteration {\em geodesic} would kick in. Once the lapse has collapsed,
{\em static} is executed, overriding the default, but {\em not}
overriding {\em geodesic}, which is still activated every second
iteration because it's positioned {\em before} the {\em static}.

\item{\bf Set active slicing} The {\em active\_slicing\_handle} is set
at {\tt CCTK\_BASEGRID} and {\tt CCTK\_PRESTEP}, {\em before} the
evolution routine.

\item{\bf Checking the for the active slicing}: Within the program,
which provides a slicing, a check for the active slicing has to be
done. Assuming a Fortran code has registered the two slicings
``geodesic'' and ``static'', the code would like this:
\begin{verbatim}
subroutine MY_EVOLUTION(CCTK_ARGUMENTS)

c     declare this function
      INTEGER Einstein_GetSlicingHandle

c     compare the handle for ``geodesic'' to active_slicing_handle
      if (active_slicing_handle .eq. 
     .    Einstein_GetSlicingHandle("geodesic")) then
        <EXECUTE GEODESIC>
      else if (active_slicing_handle .eq. 
     .         Einstein_GetSlicingHandle("static")) then
        <EXECUTE STATIC>
      else if (Einstein_GetSlicingHandle(slicing).lt.0)
         call CCTK_WARN(0,"ERROR: slicing in parfile not registered")
      endif
\end{verbatim}
Here we first check for if the active handle corresponds to ``{\em
geodesic}'', then ``{\em static}'' and if that fails, we do some
primitive error management and check if the value of slicing is
actually the key to a slicing in the database.

\end{enumerate}



% Automatically created from the ccl files by using gmake thorndoc
\include{interface}
\include{param}
\include{schedule}

%%%%%%%%%%%%%%%%%%%%%%%%%%%%%%%%%%%%%%%%%%%%%%%%%%%%%%%%%%%%%%%%%%%%%%%%%%%%%%%%
%%%%%%%%%%%%%%%%%%%%%%%%%%%%%%%%%%%%%%%%%%%%%%%%%%%%%%%%%%%%%%%%%%%%%%%%%%%%%%%%
%%%%%%%%%%%%%%%%%%%%%%%%%%%%%%%%%%%%%%%%%%%%%%%%%%%%%%%%%%%%%%%%%%%%%%%%%%%%%%%%

\end{document}
