% /*@@
%   @file      Preface.tex
%   @date      Wed Jan 12 14:37:52 2000
%   @author    Tom Goodale
%   @date      
%   
%   @enddesc 
%   @version $Header: /cactus/Cactus/doc/MaintGuide/Preface.tex,v 1.1 2000/01/12 18:01:31 goodale Exp $
% @@*/

%%%%%%%%%%%%%%%%%%%%%%%%%%%%%%%%%%%%%%%%%%%%%%%%%%%%%%%%%%%%%%%%%%%%%%%
%%%%%%%%%%%%%%%%%%%%%%%%%%%%%%%%%%%%%%%%%%%%%%%%%%%%%%%%%%%%%%%%%%%%%%%

{\large \bf Preface} 
\label{sec:pr}
 
\vskip .5cm

This document should describe the Cactus flesh.  In particular
it should describe

\begin{itemize}
\item
The philosophy of the flesh
\item
The coding style used
\item
The make system
\item
The various source directories and all their files
\item
The perl scripts
\end{itemize}

In addition it should contain ideas for future enhancements.

\vskip .5cm

{\bf Overview of documentation}

\vskip .5cm

This guide covers the following topics

\begin{Lentry}

\item [{\bf Part A: Philosphy and Style.}]
  The philosophy behind the flesh and the coding style used.

\item [{\bf Part B: The Make System.}] 
  The nitty-gritty of the make system.

\item [{\bf Part C: The CST.}] 
  The nitty-gritty of the CST.

\item [{\bf Part D: General.}]
  General miscellaneous things used all over the flesh.
 
\item [{\bf Part E: Main.}] 
  Everything you never wanted to know about the files in the Main
  subdirectory of the flesh.

\item [{\bf Part F: Comm.}] 
  Everything you never wanted to know about the files in the Comm
  subdirectory of the flesh.

\item [{\bf Part G: IO.}] 
  Everything you never wanted to know about the files in the IO
  subdirectory of the flesh.

\item [{\bf Part H: Util.}] 
  Everything you never wanted to know about the various utility
  files.

\item [{\bf Part I: Schedule.}] 
  Everything you never wanted to know about the Schedule system.

\item [\bf Part J: Appendices.] 
  I'm sure we'll need something here.

\end{Lentry}

Other topics to be discussed in separate documents include:

\begin{Lentry}

\item [{\bf Computational Thorn Guide}] This will contain details about the 
arrangements and thorns making up the standard Cactus Computation Tool Kit

\item [{\bf Relativity Thorn Guide}] This will contain details about the arrangements and thorns making up the Cactus Relativity Tool Kit, one of the major 
 motivators, and still the driving force, for the Cactus Code.

\item [{\bf Users' Guide}] 
  The stuff users need to know.  This in particular documents
  the functions the flesh needs to make available to the thorns.

\end{Lentry}

\vskip .5cm

{\bf Typographical Conventions}

\begin{Lentry}

\item[{\tt Typewriter}] Is currently used for everything you type,
	for program names, and code extracts.
\item[{\tt < ... >}] Indicates a compulsory argument.
\item[{\tt [ ... ]}] Indicates an optional argument.

\end{Lentry}
 
\vskip .5cm

{\bf How to Contact Us}

\vskip .5cm

Please let us know of any errors or omissions in this guide, as well
as suggestions for future editions. These can be reported via our 
bug tracking system at {\tt www.cactuscode.org}, or via email to
{\tt cactus@cactuscode.org}. Alternatively, write to us at

\vskip .5cm
The Cactus Team\\
Albert Einstein Institute\\
Max Planck Institute for Gravitational Physics\\
Am M\"{u}hlenberg, 1\\
Potsdam\\
Germany


\vskip .5cm

{\bf Acknowledgements}

\vskip .5cm

Hearty thanks to all those who have helped with documentation for the
Cactus Code.
