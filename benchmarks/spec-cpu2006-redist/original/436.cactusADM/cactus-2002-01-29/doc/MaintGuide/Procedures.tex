% /*@@
%   @file      Procedures.tex
%   @date      Fri May 11 11:18:04 2001
%   @author    Tom Goodale
%   @desc 
%   Various procedures
%   @enddesc 
%   @version $Header: /cactus/Cactus/doc/MaintGuide/Procedures.tex,v 1.3 2001/05/16 09:32:53 goodale Exp $
% @@*/

\begin{cactuspart}{2}{Procedures}{$RCSfile: Procedures.tex,v $}{$Revision: 1.3 $}
\renewcommand{\thepage}{\Alph{part}\arabic{page}}

%%%%%%%%%%%%%%%%%%%%%%%%%%%%%%%%%%%%%%%%%%%%%%%%%%%%%%%%%%%%%%%%%%%%%%%

\chapter{Use of CVS} 

Version control in Cactus is maintained by the use of the CVS
software.  This software allows one to trace any change to a file from 
the creation of a file to the present version, and provides an
automatic notification system to alert interested parties of changes
to files.  In order to make effective use of the system, the following 
commit procedure is recommended as a guideline

\begin{Lentry}
\item[{\em Only make one change at a time}]
Don't make a commit which changes several distinct things at once, as
it is difficult then for people tracing changes back to distinguish
which bit was changed for which reason.  See the note on commit
messages below.
\item[{\em Run the test suite}]
This makes sure the code compiles, runs, and produces the correct results. 
\item[{\em Know which files you are going to commit}]
Always check what you are about to commit by use of the

{\tt cvs -n -q update} 

command.  This ensures that you know which files have been modified,
which files have been removed and which files have been added, and
provides a useful reminder to use the {\tt \verb.cvs add.} and {\tt
\verb.cvs remove.} commands.
\item[{\em Know what has changed}]
The use of the

{\tt cvs diff} 

command on each modified file is a good check that you are not just
committing an accidental keystroke or a debug statement.  Moreover it
is a good reminder of what has changed and needs to be mentioned in
the commit message.
\item[{\em Provide clear and meaningful and relevant commit messages}]
The commit message should explain what has changed and why, for
details people can use {\tt \verb.cvs diff.}, however the commit
message should be clear enough for people to have a good idea of what
is going on.  This is strongly coupled to the item about making only
one change listed above - if two distinct things have been changed,
they should be committed separately, with relevant commit messages.
If the change resulted from a {\tt Problem Report} (PR) the PR number
should be noted in the commit message.
\end{Lentry}

%%%%%%%%%%%%%%%%%%%%%%%%%%%%%%%%%%%%%%%%%%%%%%%%%%%%%%%%%%%%%%%%%%%%%%%

\chapter{Use of GNATS}

Bug tracking in Cactus is maintained by use of the GNATS software.
This software provides audit trails of the status and all
correspondence concerning any problem report (PR).  Each problem is
given a unique number and assigned a responsible person.

\begin{Lentry}
\item[{\em Correspondence}]
All correspondence with the author of the PR should be copied to {\tt
\verb,bugs@cactuscode.org,} with the subject line assigned by Gnats.
This ensures that the correspondence is entered into the audit trail.
\item[{\em Responsiblity}]
When a PR comes in, it is assigned a responsible person.  If another
person wishes to tackle the problem they should check with the
responsible person, and then assign themselves as the responsible
person.
\item[{\em Initial auditing}]
The responsible person should review the PR and check that the user
supplied fields are sensible.  I particular the {\tt Synopsis} should
be an accurate reflection of the problem, and the {\tt Priority} and
{\tt Severity} fields should be set to the correct levels.  If the
{\tt Release} field is badly filled out, attempts should be made to
determine the relase version used by the PR submitter.  If it is a
duplicate of a previous PR it should be marked as {\tt Duplicate}.
\item[{\em Analysing the PR}]
Once the responsible person has had a chance to review the PR, they
should either seek further information from the submitter and mark the 
PR state as {\tt Feedback}, or they should seek to determine the cause 
of the problem and mark it as {\tt Analysed}.
\item[{\em Closing a PR}]
Once a problem is fixed, the PR state should be changed to the current 
version number of Cactus.  The {\tt Fix} field should be filled out
with what was done, and {\tt CVS} version numbers for the changed
files should be noted.  Any miscellaneous comments about the problem
should be noted in the {\tt Release-Note} field.
\end{Lentry}

%%%%%%%%%%%%%%%%%%%%%%%%%%%%%%%%%%%%%%%%%%%%%%%%%%%%%%%%%%%%%%%%%%%%%%%

\chapter{Release procedure}

In the beta release cycle, Cactus is maintained in two CVS
repositories - the `stable' \verb./cactus. and the development
\verb./cactusdevcvs..  The stable version is the last beta release and 
no commits should ever be made to it - it is for people who do not
want to worry about things breaking from day to day.  The development
version is the tree used for developing the next beta release.

Making a beta release consists of copying the cvs modules from the
development repository to the stable repository.  The following
procedure is used:

\begin{Lentry}
\item[{\em Notify committers of start of release procedure}]
This ensures that no commits are made during the following procedure.
If it is impossible, for some reason, to notify a person of the start
of the procedure, that person's commit rights should be revoked during 
the procedure to prevent accidents.
\item[{\em Check the code on all supported architectures}]
The code should be checked out (in a fresh place), compiled and the
test-suites run on all suppported architectures.  Problems found
should be fixed or noted in the release notes.  This is an iterative
procedure, as any commits made to fix problems need to be checked on
all other architectures.
\item[{\em Tag the code}]
Tag the code with the latest release tag and update the {\tt LATEST}
and {\tt STABLE} tags.  The easiest way to do this is from a clean
checkout.
\begin{verbatim}
cd /Cactus
cvs -q update -d  # To check for any missing/not up-to-date files.
cvs tag Cactus_4_0_Beta_X
cvs tag -F LATEST
cvs tag -F STABLE
\end{verbatim}

\item[{\em Log into cvs machine as cactus\_admin}]
All repository maintanence should be done as the cactus admin user.
\item[{\em Store old module files}]
A directory should be made in the stable repository, and all
releaseable modules should be moved into that directory.  This will
temporarily break checkouts/updates from that repository.  A suggested 
command is of the form

\begin{verbatim}
cd /cactus
mkdir betaX
mv Cactus{,Base,Connect,Einstein,X,X,X} betaX
\end{verbatim}

\item[{\em Copy new module files}]
e.g.
\begin{verbatim}
cp -r /cactusdevcvs/Cactus{,Base,Connect,Einstein,X,X,X} .
\end{verbatim}
\item[{\em Fix permissions on new module files}]
\begin{verbatim}
find Cactus{,Base,Connect,Einstein,X,X,X} -type d -exec chmod 777 \;
Setperms.pl public Cactus{,Base,Connect,Einstein,X,X,X}
\end{verbatim}
\item[{\em Update CVS {\tt modules} file for new modules}]
The stable repository's {\tt modules} file should be updated with any
new module information added. 
\item[{\em Check that checkout/update works}]
A fresh checkout should be made as a double check that all permissions 
have been set correctly.
\item[{\em Update version of development tree}]
Update the version in {\tt Makefile} and commit it to the development
tree.
\item[{\em Re-enable commit access}]
People whose commit access was removed in the first part of this
procedure should have access re-enabled.
\item[{\em Notify people}]
The relase notes should be sent out to the cactus mailing lists and
any other relevant places such as linux-announce and Freshmeat.
\item[{\em Update web page}]
The release should be noted in the news section of the web page.
Additionally the html versions of the documentation should be updated.
\item[{\em Close PRs}]
Any problem reports which were closed in the beta relase should be
audited for correct entries in the {\tt Fix} field and then their
state should be marked as {\tt closed}.
\end{Lentry}

%%%%%%%%%%%%%%%%%%%%%%%%%%%%%%%%%%%%%%%%%%%%%%%%%%%%%%%%%%%%%%%%%%%%%%%

\end{cactuspart}
