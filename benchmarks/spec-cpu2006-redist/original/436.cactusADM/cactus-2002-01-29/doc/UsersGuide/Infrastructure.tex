% /*@@
%   @file      Infrastructure.tex
%   @date      27 Jan 1999
%   @author    Tom Goodale, Gabrielle Allen, Gerd Lanferman
%   @desc 
%   Infrastructure thorn writer's guide for the Cactus User's Guide
%   @enddesc 
%   @version $Header: /cactus/Cactus/doc/UsersGuide/Infrastructure.tex,v 1.28 2001/12/28 21:21:41 tradke Exp $      
% @@*/
\begin{cactuspart}{4}{Infrastructure Thorn Writer's Guide}{$RCSfile: Infrastructure.tex,v $}{$Revision: 1.28 $}
\renewcommand{\thepage}{\Alph{part}\arabic{page}}

\chapter{Introduction} 

\begin{itemize}
 \item{} Concepts and terminology (Overloading and registration of functions)
 \item{} The cGH structure --- what it is and how to use it
 \item{} Extending the cGH structure
 \item{} Querying group and variable information
 \item{} Providing an IO layer
 \item{} Providing a communication layer
 \item{} Providing a reduction operator
 \item{} Providing an interpolation operator
 \item{} Overloadable functions
\end{itemize}

\chapter{Concepts and Terminology}
\label{chap:cote}

\section{Overloading and Registration}

The flesh defines a core API which guarantees the presence of a set of 
functions.  Although the flesh guarantees the presence of these functions,
they can be provided by thorns.  Thorns do this either by the {\em overloading} 
or the {\em registration} of functions.

\subsection{Overloading}

Some functions can only be provided by one thorn.  The first thorn to
{\em overload} this function succeeds, and any later attempt to overload
the function fails.  For each overloadable function there is a function
with a name something like {\tt CCTK\_Overload...} which is passed the
function pointer.

\subsection{Registration}

Some functions may be provided by several thorns.  The thorns {\em register}
their function with the flesh, and when the flesh-provided function is called, 
the flesh calls all the registered functions.

\section{GH Extensions}

A GH extension is a way to associate data with each cGH.  This data should be data 
that is required to be associated with a particular GH by a thorn.

Each GH extension is given a unique handle.

\section{IO Methods}

An IO method is a distinct way to output data.  Each IO method has a unique name,
and the flesh-provided IO functions operate on all registered IO methods.

\chapter{GH Extensions}

A GH extension is created by calling {\tt CCTK\_RegisterGHExtension}, with the 
name of the extension.  This returns a unique handle that identifies the extension.
(This handle can be retrieved at any time by a call to {\tt CCTK\_GHExtensionHandle}.)

Associated with a GH extension are three functions

\begin{Lentry}
\item[SetupGH]
this is used to actually create the data structure holding the extension.  It
is called when a new cGH is created.
\item[InitGH]
this is used to initialise the extension.  It is called after the scheduler has
been initialised on the cGH.
\item[ScheduleTraverseGH]
this is called whenever the schedule tree is due to be traversed on the GH.  It
should initialise the data on the cGH and the call {\tt CCTK\_ScheduleTraverse} to traverse 
the schedule tree.
\end{Lentry}

\chapter{IO Methods}
\label{chap:iome}

\chapter{Overloadable and Registerable Functions in Main}

 \begin{tabular}{|l|l|}
   \hline  {\bf Function} & {\bf Default} \\
   \hline {\t CCTK\_Initialise}           &\\
   \hline {\t CCTK\_Evolve}               &\\
   \hline {\t CCTK\_Shutdown}             &\\
   \hline
 \end{tabular}

\chapter{Overloadable and Registerable Functions in Comm}

  \begin{tabular}{|l|l|}
   \hline {\bf Function} & {\bf Default}  \\
   \hline {\t CCTK\_SyncGroup}           &\\
   \hline {\t CCTK\_EnableGroupStorage}  &\\
   \hline {\t CCTK\_DisableGroupStorage} &\\
   \hline {\t CCTK\_EnableGroupComm}     &\\
   \hline {\t CCTK\_DisableGroupComm}    &\\
   \hline {\t CCTK\_Barrier}             &\\
   \hline {\t CCTK\_Reduce}              &\\
   \hline {\t CCTK\_Interp}              &\\
   \hline {\t CCTK\_ParallelInit}        &\\
   \hline
  \end{tabular}

\chapter{Overloadable and Registerable Functions in IO}

 \begin{tabular}{|l|l|} 
   \hline {\bf Function} & {\bf Default}   \\
   \hline {\t CCTK\_OutputGH}            & \\
   \hline {\t CCTK\_OutputVarAsByMethod} & \\
   \hline
 \end{tabular}

\chapter{Adding a Driver}

The flesh knows nothing about memory allocation for grid variables, or about how
to communicate data when synchronisation is called for.  It knows nothing about 
multiple patches or adaptive mesh refinement.  All this is the job of a driver.

\section{Anatomy}

A driver consists of a Startup routine which creates a GH extension and 
registers its associated functions, and overloads the communication functions.
It may optionally register interpolation, reduction, and IO methods.

A driver may also overload the default Initialisation and Evolution routines,
although a simple unigrid evolver is supplied in the flesh.

\section{Startup}

A driver consists of a GH extension, and the following overloaded
functions.

\begin{enumerate}
\item{} CCTK\_EnableGroupStorage
\item{} CCTK\_DisableGroupStorage
\item{} CCTK\_ArrayGroupSizeB
\item{} CCTK\_QueryGroupStorageB
\item{} CCTK\_SyncGroup
\item{} CCTK\_EnableGroupComm
\item{} CCTK\_DisableGroupComm
\item{} CCTK\_Barrier
\item{} CCTK\_OverloadParallelInit
\item{} CCTK\_OverloadExit
\item{} CCTK\_OverloadAbort
\item{} CCTK\_OverloadMyProc
\item{} CCTK\_OverloadnProcs
\end{enumerate}

\section{The GH Extension}

The GH extension is where the driver stores all its grid-dependent information.
This is stuff like any data associated with a grid variable (e.g. storage and
communication state), how many grids if it is AMR, ...  It is very difficult to 
describe in general, but one simple example might be

\begin{verbatim}

struct SimpleExtension
{
  /* The data assocatiated with each variable */
  /* data[var][timelevel][ijk]                */
  void ***data
} ;

\end{verbatim}

with a SetupGH routine like.

\begin{verbatim}

struct SimpleExtension *SimpleSetupGH(tFleshConfig *config, int conv_level, cGH *GH)
{
   struct SimpleExtension *extension;

   extension = NULL;

   if(conv_level < max_conv_level)
   {  
      /* Create the extension */
      extension = malloc(sizeof(struct SimpleExtension));

      /* Allocate data for all the variables */
      extension->data = malloc(num_vars*sizeof(void**));

      for(var = 0 ; var < num_vars; var++)
      {
        /* Allocate the memory for the time levels */
        extension->data[var] = malloc(num_var_time_levels*sizeof(void *));

        for(time_level = 0; time_level < num_var_time_level; time_level++)
        {
          /* Initialise the data to NULL */ 
          extension->data[var][time_level] = NULL;
        }
      }
    }

   return extension;
}

\end{verbatim}

Basically all this example is doing is preparing a data array for use.  The
function can query the flesh for information on every variable.  Note that 
scalars should always have memory actually assigned to them.

An {\tt InitGH} function isn't strictly necessary, and in this case it could just
be a dummy function.

The {\tt ScheduleTraverseGH} function needs to fill out the cGH data and then call 
{\tt CCTK\_ScheduleTraverse} to have the functions scheduled at that point executed on the
grid.

\begin{verbatim}

int SimpleScheduleTraversGH(cGH *GH, const char *where)
{
  int retcode;
  int  var;
  int  gtype;
  int  ntimelevels;
  int  level;
  int  idir;

  extension = (struct SimpleExtension *)GH->extensions[SimpleExtension];

  for (idir=0;idir<GH->cctk_dim;idir++)
  {
    GH->cctk_levfac[idir] = 1;
    GH->cctk_nghostzones[idir] = extension->nghostzones[idir];
    GH->cctk_lsh[idir]         = extension->lnsize[idir];
    GH->cctk_gsh[idir]         = extension->nsize[idir];
    GH->cctk_bbox[2*idir]      = extension->lb[extension->myproc][idir] == 0;
    GH->cctk_bbox[2*idir+1]    = extension->ub[extension->myproc][idir] 
                              == extension->nsize[idir]-1;
    GH->cctk_lbnd[idir]        = extension->lb[extension->myproc][idir];
    GH->cctk_ubnd[idir]        = extension->ub[extension->myproc][idir];
    
  }

  for(var = 0; var < extension->nvariables; var++)
  {
    gtype = CCTK_GroupTypeFromVarI(var);
    ntimelevels = CCTK_NumTimeLevelsFromVarI(var);

    for(level = 0; level < ntimelevels; level++)
    {
      switch(gtype)
      {
        case CCTK_SCALAR :
          GH->data[var][level] = extension->variables[var][level]; 
          break;
        case CCTK_GF     :
          GH->data[var][level] = 
            ((pGF ***)(extension->variables))[var][level]->data; 
          break;
        case CCTK_ARRAY :
          GH->data[var][level] = 
            ((pGA ***)(extension->variables))[var][level]->data; 
          break;
        default:
          CCTK_WARN(1,"Unknown group type in SimpleScheduleTraverse");
      }
    }
  }

  retcode = CCTK_ScheduleTraverse(where, GH, NULL);

  return retcode;

}

\end{verbatim}

The third argument to {\tt CCTK\_ScheduleTraverse} is actually a function
which will be called by the schedular when it wants to call a function
scheduled by a thorn.  This function is given some information about
the function to call, and is an alternative place where the cGH can be setup.

This function is optional, but a simple implementation might be

\begin{verbatim}

int SimpleCallFunction(void *function, 
                       cFunctionData *fdata, 
                       void *data)
{
  void (*standardfunc)(void *);

  int (*noargsfunc)(void);

  switch(fdata->type)
  {
    case FunctionNoArgs:
      noargsfunc = (int (*)(void))function;
      noargsfunc();
      break;
    case FunctionStandard:
      switch(fdata->language)
      {
        case LangC:
          standardfunc = (void (*)(void *))function;
          standardfunc(data);
          break;
        case LangFortran:
          fdata->FortranCaller(data, function);
          break;
        default :
          CCTK_WARN(1, "Unknown language.");
      }
      break;
    default :
      CCTK_WARN(1, "Unknown function type.");
  }

  /* Return 0, meaning didn't synchronise */
  return 0;
}

\end{verbatim}

The return code of the function signifies whether or not the function
synchronised the groups in this functions synchronisation list of not.

The flesh will synchronise them if the function returns false.

Providing this function is probably the easiest way to do multi-patch or
AMR drivers.

\section{Memory Functions}

These consist of 
\begin{enumerate}
\item{} CCTK\_EnableGroupStorage
\item{} CCTK\_DisableGroupStorage
\item{} CCTK\_QueryGroupStorageB
\item{} CCTK\_ArrayGroupSizeB
\end{enumerate}

\subsection{En/Disable Group Storage}

These are responsible for switching the memory for all variables
in a group on or off.  They should return the former state, e.g.
if the group already has storage assigned, they should return 1.

In our simple example above, the enabling routine would look 
something like

\begin{verbatim}

int SimpleEnableGroupStorage(cGH *GH, const char *groupname)
{

  extension = (struct SimpleExtension *)GH->extensions[SimpleExtension];

  if(extension->data[first][0][0] == NULL)
  {
    for(var = first; var <= last; var++)
    {
      allocate memory for all time levels;
    }
    retcode = 0;
  }
  else
  {
    retcode = 1;
  }

  return retcode;
}

\end{verbatim}

The disable function is basically the reverse of the enable one.

The QueryGroupStorage function basically returns true or false if there is storage
for the group, and the ArrayGroupSize returns the size of the grid function or array
group in a particular direction.

\subsection{En/Disable Group Comm}

These are the communication analogues to the storage functions.  Basically they flag
that communication is to be done on that group or not, and may initialise data 
structures for the communication.



\chapter{Adding an I/O Method}
\label{chap:io_methods}
%
The flesh does not implement output for grid variables and other data by itself.
Instead it provides a mechanism for thorns to register their own
routines as I/O methods, and to invoke these I/O methods by either the
flesh scheduler or by other thorn routines.
%
\section{I/O Method Registration}
%
All I/O methods have to be registered with the flesh before they can be used.\\
The flesh I/O registration API provides a routine {\t CCTK\_RegisterIOMethod()}
to create a handle for a new I/O method which is identified by its name
(this name must be unique for all I/O methods).
With such a handle, a thorn can then register a set of functions (using the
{\t CCTK\_RegisterIOMethod*()} routines from the flesh) for doing
periodic, triggered, and/or unconditional output.

The following example shows how a thorn would register an I/O method {\it
"SimpleIO"} with routines to provide all these different types of output.
%
\begin{verbatim}
  void SimpleIO_Startup (void)
  {
    int handle;


    handle = CCTK_RegisterIOMethod ("SimpleIO");
    if (handle >= 0)
    {
      CCTK_RegisterIOMethodOutputGH (handle, SimpleIO_OutputGH);

      CCTK_RegisterIOMethodTimeToOutput (handle, SimpleIO_TimeToOutput);
      CCTK_RegisterIOMethodTriggerOutput (handle, SimpleIO_TriggerOutput);

      CCTK_RegisterIOMethodOutputVarAs (handle, SimpleIO_OutputVarAs);
    }
  }
\end{verbatim}
%
%
\section{Periodic Output of Grid Variables}
%
The flesh scheduler will automatically call {\t CCTK\_OutputGH()} at every
iteration after the CCTK\_ANALYSIS time bin. This function loops over all I/O
methods and calls their routines registered as {\t OutputGH()} (see also
\ref{subsec:schedule_ccl}).
%
\begin{verbatim}
  int SimpleIO_OutputGH (const cGH *GH);
\end{verbatim}
%
The {\t OutputGH()} routine itself should loop over all variables living on the
{\t GH} grid hierarchy, and do all neccessary output if requested (this is
usually determined by I/O parameter settings).

As its return code it should pass back the number of variables which were output
at the current iteration, or zero if no output was done by this I/O method.
%
%
\section{Triggered Output of Grid Variables}
%
Besides the periodic output at every so many iterations using {\t OutputGH()},
analysis and output of grid variables can also be done via triggers. For this,
a {\t TimeToOutput()} routine is registered with an I/O method.
This routine will be called by the flesh scheduler at every iteration before
CCTK\_ANALYSIS with the triggering variable(s) as defined in the schedule
block for all CCTK\_ANALYSIS routines (see \ref{scheduling:schedule_block}).

If the {\t TimeToOutput()} routine decides that it is now time to do output, the
flesh scheduler will invoke the corresponding analysis routine and also request
output of the triggering variable(s) using {\t TriggerOutput()}.
%
\begin{verbatim}
  int SimpleIO_TimeToOutput (const cGH *GH, int varindex);
  int SimpleIO_TriggerOutput (const cGH *GH, int varindex);
\end{verbatim}
%
Both routines get passed the index of a possible triggering grid variable.

{\t TimeToOutput()} should return true (a non-zero value) if analysis and output
for {\it varindex} should take place at the current iteration, and false (zero)
otherwise.\\
{\t TriggerOutput()} should return zero for successful output of variable
{\it varindex}, and a negative value in case of an error.
%
%
\section{Unconditional Output of Grid Variables}

An I/O method's {\t OutputVarAs()} routine is supposed to do output for a
specific grid variable if ever possible. It will be invoked by the flesh I/O API
routines {\t CCTK\_OutputVar*()} for unconditional, non-triggered output of
grid variables (see also \ref{sec:io}).

A function registered as an {\t OutputVarAs()} routine has the following
prototype:
%
\begin{verbatim}
  int SimpleIO_OutputVarAs (const cGH *GH, const char *varname, const char *alias);
\end{verbatim}
%
The variable to output, {\it varname}, is given by its full name.
In addition to that, an {\it alias} string can be passed which then serves
the purpose of constructing a unique name for the output file.

The {\t OutputVarAs()} routine should return zero if output for {\it varname}
was done successfully, or a negative error code otherwise.


\chapter{Adding a timer}

To add a Cactus timer you need to write several functions to provide the 
timer functionality, and then register these functions with the flesh, with a name for the timer using {\t CCTK\_TimerRegister}.

The functions are registered using a structure {\t cTimerFuncs} which
contains the function pointers. The required functions are:

\begin{Lentry}

\item[{\t info.n\_vals}]
\item[{\t create}] {\t void *(*create)(int);}
\item[{\t destroy}] {\t void (*destroy)(int, void *)}
\item[{\t start}] {\t void (*start)(int, void *)}
\item[{\t stop}] {\t void (*stop)(int, void *)}
\item[{\t reset}] {\t void (*reset)(int, void *)}
\item[{\t get}] {\t void (*get)(int, void *, cTimerVal *)}
\item[{\t set}] {\t void (*set)(int, void *, cTimerVal *)}

\end{Lentry}



%%%%%%%%%%%%%%%%%%%%%%%%%%%%%%%%%%%%%%%%%%%%%%%%%%%%%%%%%%%%%%%%%%%%%%%%%%
\end{cactuspart}
