% /*@@
%   @file      Preface.tex
%   @date      27 Jan 1999
%   @author    Tom Goodale, Gabrielle Allen, Gerd Lanferman
%   @desc 
%   Preface for the Cactus User's Guide
%   @enddesc 
%   @version $Header: /cactus/Cactus/doc/UsersGuide/Preface.tex,v 1.7 2001/12/17 10:06:13 rideout Exp $      
% @@*/

%%%%%%%%%%%%%%%%%%%%%%%%%%%%%%%%%%%%%%%%%%%%%%%%%%%%%%%%%%%%%%%%%%%%%%%
%%%%%%%%%%%%%%%%%%%%%%%%%%%%%%%%%%%%%%%%%%%%%%%%%%%%%%%%%%%%%%%%%%%%%%%

{\large \bf Preface} 
\label{sec:pr}
 
\vskip .5cm

This document will eventually be a complete guide to using and
developing with the Cactus Code. However, it is currently under
development, so please be patient if you can't find what you need.
Please report omissions, errors, or suggestions to 
and of our contact addresses below, and we will
try and fix them as soon as possible. 

\vskip .5cm

{\bf Overview of documentation}

\vskip .5cm

This guide covers the following topics

\begin{Lentry}

\item [{\bf Part A: Installation and Running.}]
  We give an overview of required hardware and
  software and talk you through the installation of a working
  Cactus tool kit. 
  You will be able to verify the correct installation with our
  test suite technology. 

\item [{\bf Part B: Application Thorn Writing.}] We introduce thorn
  concepts and describe all you need to know for creating, writing
  and maintaining application thorns. You will learn how to use the
  programming interface to take advantage of parallelism and modularity.

\item [{\bf Part C: Infrastructure Thorn Writer's Guide.}] In this more
  advanced part, we talk about user supplied infrastructure routines such
  as {\em additional output routines, drivers, etc.} [This part is not
  yet complete, and is currently under construction.]

\item [{\bf Part D: Function Reference.}] Here all Cactus flesh functions which
are available to thorn writers in C and Fortran are described.

\item [\bf Part E: Appendices.] These contain a description of the
Cactus Configuration Language, and other odds and ends, such as how to
use GNATS or TAGS.

\end{Lentry}

Other topics to be discussed in separate documents include:

\begin{Lentry}

\item [{\bf Computational Thorn Guide}] This will contain details about the 
arrangements and thorns making up the standard Cactus Computational Tool Kit.

\item [{\bf Relativity Thorn Guide}] This will contain details about the arrangements and thorns making up the Cactus Relativity Tool Kit, one of the major 
 motivators, and still the driving force, for the Cactus Code.

\item [{\bf Flesh Maintainers Guide}] 
 This will contain all the gruesome details
 about the inner workings of Cactus, for all those who want or need to 
 expand or maintain the core of Cactus.

\end{Lentry}

\vskip .5cm

{\bf Typographical Conventions}

\begin{Lentry}

\item[{\tt Typewriter}] Is currently used for everything you type,
	for program names, and code extracts.
\item[{\tt < ... >}] Indicates a compulsory argument.
\item[{\tt [ ... ]}] Indicates an optional argument.

\end{Lentry}
 
\vskip .5cm

{\bf How to Contact Us}

\vskip .5cm

Please let us know of any errors or omissions in this guide, as well
as suggestions for future editions. These can be reported via our 
bug tracking system at {\tt www.cactuscode.org}, or via email to
{\tt cactus@cactuscode.org}. Alternatively, write to us at

\vskip .5cm
The Cactus Team\\
Albert Einstein Institute\\
Max Planck Institute for Gravitational Physics\\
Am M\"{u}hlenberg 1\\
D-14476 Golm\\
Germany


\vskip .5cm

{\bf Acknowledgements}

\vskip .5cm

Hearty thanks to all those who have helped with documentation for the
Cactus Code. Special thanks to those who struggled with the earliest
sparse versions of this guide and sent in mistakes and suggestions,
in particular John Baker, Carsten Gundlach, Ginny Hudak-David, 
Sai Iyer, Paul Lamping, Nancy Tran and Ed Seidel. 

