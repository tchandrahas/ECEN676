\par
\section{Driver programs for the {\tt Coords} object}
\label{section:Coords:drivers}
\par
This section contains brief descriptions of the driver programs.
\par
%=======================================================================
\begin{enumerate}
%-----------------------------------------------------------------------
\item
\begin{verbatim}
testIO msglvl msgFile inFile outFile
\end{verbatim}
This driver program reads and write {\tt Coords} files, useful for
converting formatted files to binary files and vice versa.
One can also read in a {\tt Coords} file and print out just the 
header information (see the {\tt Coords\_writeStats()} method).
\par
\begin{itemize}
\item
The {\tt msglvl} parameter determines the amount of output ---
taking {\tt msglvl >= 3} means the {\tt Coords} object is written
to the message file.
\item
The {\tt msgFile} parameter determines the message file --- if {\tt
msgFile} is {\tt stdout}, then the message file is {\it stdout},
otherwise a file is opened with {\it append} status to receive any
output data.
\item
The {\tt inFile} parameter is the input file for the {\tt Coords}
object. It must be of the form {\tt *.coordsf} or {\tt *.coordsb}.
The {\tt Coords} object is read from the file via the
{\tt Coords\_readFromFile()} method.
\item
The {\tt outFile} parameter is the output file for the {\tt Coords}
object. 
If {\tt outFile} is {\tt none} then the {\tt Coords} object is not
written to a file. 
Otherwise, the {\tt Coords\_writeToFile()} method is called to write
the object to 
a formatted file (if {\tt outFile} is of the form {\tt *.coordsf}),
or
a binary file (if {\tt outFile} is of the form {\tt *.coordsb}).
\end{itemize}
%-----------------------------------------------------------------------
\item
\begin{verbatim}
mk9PCoords msglvl msgFile n1 n2 outCoordsFile
\end{verbatim}
This driver program creates a {\tt Coords} object 
for 9-point finite difference operator on 
a $\mbox{\tt n1} \times \mbox{\tt n2}$ grid
and optionally writes it to a file.
\begin{itemize}
\item
The {\tt msglvl} parameter determines the amount of output ---
taking {\tt msglvl >= 3} means that all objects are written
to the message file.
\item
The {\tt msgFile} parameter determines the message file --- if {\tt
msgFile} is {\tt stdout}, then the message file is {\it stdout},
otherwise a file is opened with {\it append} status to receive any
message data.
\item
The {\tt outCoordsFile} parameter is the output file for the 
{\tt Coords} object. 
If {\tt outCoordsFile} is {\tt none} then the {\tt Coords} object 
is not written to a file. 
Otherwise, the {\tt Coords\_writeToFile()} method is called to write
the object to 
a formatted file (if {\tt outCoordsFile} 
is of the form {\tt *.coordsf}), or
a binary file (if {\tt outCoordsFile} is of the form {\tt *.coordsb}).
\end{itemize}
%-----------------------------------------------------------------------
\end{enumerate}
